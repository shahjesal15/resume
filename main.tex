\documentclass{resume} 
\usepackage[left=0.6in,top=0.6in,right=0.8in,bottom=0.6in]{geometry} 
\pagenumbering{arabic}
\usepackage{setspace}

\usepackage{fontawesome}
\usepackage{enumitem}
\usepackage{datetime2}
% \usepackage{xcolor}
% \renewcommand\thefootnote{\textcolor{black}{\arabic{footnote}}}
\usepackage[symbol]{footmisc}
\usepackage{textcomp}
\newcommand{\smallspace}{\hspace{0.2em}}

\newcommand{\resumeHeadingListStart}{
  \begin{itemize}[leftmargin=0.15in, label={}]
    
}
% \newcommand*{\subsectionstyles}[1]{{\fontsize{10pt}{1em}\bodyfont\textcolor{text}{#1}}}
\newcommand{\resumeHeadingListEnd}{\end{itemize}}
\newcommand{\resumeSectionType}[3]{
  \hspace{-1.3em}\item\begin{tabular*}{0.5\textwidth}[t]{
    p{0.25\linewidth}p{0.02\linewidth}p{0.67\linewidth}
  }
  \normalsize
    \textbf{#1} & #2 & {#3}
  \end{tabular*}\vspace{-8pt}
}

\usepackage{fancyhdr}
\pagestyle{fancy}
\fancypagestyle{firstpage}
{
    \fancyhead[L]{}    
    % \fancyfoot[R]{\color{date} Last Updated on \today}
}
\fancypagestyle{reference}
{
    \fancyfoot[L] {}
    \fancyfoot[C] {}
    \fancyfoot[R] {\color{date} \thepage}
}
\fancyhead{}
\fancyfoot{}
\renewcommand{\headrulewidth}{0pt}
\renewcommand{\footrulewidth}{0pt}

\fancyfoot[C] {\color{date} \thepage}
\fancyfoot[R]{\color{date} Last Updated on \today}
% 483d8b, 8000000
\usepackage[dvipsnames]{xcolor}
\definecolor{hypercolor}{HTML}{800000}
\definecolor{date}{HTML}{666666} 
\usepackage{hyperref}
\usepackage{academicons}
\hypersetup{
    colorlinks=true,
    urlcolor=hypercolor,
}

% \usepackage{mathpazo} 
\usepackage{fontspec}
% \setmainfont[
% BoldFont = Source Sans Pro-{Semi-bold}]{Source Sans Pro}
\setmainfont{SourceSansPro-Regular.otf}[
  Path = fonts/,
  BoldFont = SourceSansPro-SemiBold.otf,
  ItalicFont = SourceSansPro-It.otf,
  BoldItalicFont = SourceSansPro-BoldIt.otf
]

% \setromanfont[Mapping={tex-text}, 
% 	Numbers={OldStyle},
% 	Ligatures={Common}]%{Minion Pro}
%     {Caladea}
% \setmainfont[
% BoldFont = Open Sans {Semi-bold}]{Open Sans}
% \setsansfont[
% BoldFont = Roboto-Medium]{Roboto}

% Caladea
\name{Jesal Deepakbhai Shah}
\address{\hspace{0.2em} \hspace{0.4em} {\small \faEnvelope} \hspace{0.1em} \href{mailto:jesalshah1510@gmail.com}{jesalshah1510@gmail.com} \hspace{0.4em}
{\large\faGithub} \href{https://github.com/shahjesal15}{Jesal Shah} \hspace{0.4em} 
{\large\faHome} \href{https://jesalshah.github.io/}{jesalshah} \hspace{0.4em} 
{\large\faLinkedin} \href{https://linkedin.com/in/shahjesal15}{\texttt{shahjesal15}}} 
\begin{document}
% \setstretch{1.1}
\fancyheadoffset[RO]{5em}
\fancyfootoffset[RO]{0cm}
\thispagestyle{firstpage}
%----------------------------------------------------------------------------------------
%	EDUCATION 
%----------------------------------------------------------------------------------------
\vspace{-10pt}
\begin{rSection}{Education}
%--copy and paste this region  if you need more--


{\textbf {B.E. in Electronics and Communication}} \hfill {CGPA: 7.63/10.0} \\
{\href{https://www.gtu.ac.in/}{Gujarat Technological University}} \hfill {Jun 2024} 
\begin{itemize}
    % Thesis
  \item[] \vspace{-0.5em} \hspace{-1.0em} Thesis: \emph{Impedance-based trajectory tracking for quadrupedal locomotion}
  % Focus
  \item[] \vspace{-0.5em} \hspace{-1.0em} Focus: \emph{Robot Dynamics}, \smallspace \emph{Legged Locomotion}, \smallspace \emph{Computed Torque Control}, \smallspace and \smallspace \emph{Robotics Software}.
\end{itemize}
\end{rSection}
%----------------------------------------------------------------------------------------
%	PROFESSIONAL Experience
%----------------------------------------------------------------------------------------
\begin{rSection}{Professional experience}
% -- Strider Robotics full-time --
{\bf Associate Robotics Engineer} \hfill {\ Sept 2024 -- Present }\\ 
{\href{https://www.strider-robotics.in/}{Strider Robotics}} \hfill {\ {Bengaluru, IN}}
\begin{cvitems}
  \item Performed system identification and developed actuator models using machine and reinforcement learning, enhancing sim-to-real transfer of reinforcement learning policies for a quadrupedal hardware.  
  \item Designed and implemented push- and fall-recovery behaviors for a quadrupedal robot to recover from slips, external pushes, and full fall scenarios.
  \item Improved Model Predictive Control (MPC) and implemented a foothold optimization framework, achieving more stable locomotion for a quadrupedal robot on uneven terrain.
  \item Designed and implemented a Kalman Filter–based state estimator fusing IMU, leg kinematics, and contact data; validated against motion-capture ground truth, resulting in highly accurate and robust state estimation for a quadrupedal robot.
  \item Maintained and extended the robot software stack and developed an internal SDK, enabling faster experimentation and scalable development for control and navigation workflows.
\end{cvitems}
%--Strider Robotics Intern--
{\bf  Robotics Intern} \hfill {\ Jan 2024 -- Aug 2024 }\\ 
{\href{https://www.strider-robotics.in/}{Strider Robotics}} \hfill {\ {Bengaluru, IN}}
\begin{cvitems}
  \item Developed an impedance-based trajectory tracker for quadrupedal legs within the MPC framework using computed torque control algorithm, enabling force-compliant behavior in leg motion.
  \item Benchmarked multiple actuator designs, including quasi-direct drive and harmonic drives, evaluating speed, torque, and efficiency, and provided quantitative data for optimal actuator selection.
  \item Integrated the Unitree SDK for hardware testing, enabling seamless deployment and evaluation of control algorithms on the GO1 robot.
\end{cvitems}
%--Blog writer--
{\bf  Technical Content Writer} \hfill {\ Jan 2023 -- Present }\\ 
{\href{https://www.circuitbread.com/tutorials/series/communication-protocols}{CircuitBread}} \hfill {\ ({Remote}) Boise, US}
\begin{cvitems}
    \item Published in-depth articles on communication protocols for embedded systems, contributing to an online learning platform used by students and early-career engineers
    \item Collaborate with the CircuitBread team on ongoing educational content, contributing to documentation and future technical tutorials.
\end{cvitems}
\end{rSection}

%--------------------------------------------------------------------------------
%    Leadership and Activities 
%-----------------------------------------------------------------------------------------------
\begin{rSection}{Leadership and Activities}
%-- Software Lead --
{\bf Software Lead}  \hfill {\ Aug 2022 -- Nov 2023}\\
{\href{https://www.gturoboticsclub.com/acheivements}{GTU Robotics Club}} \hfill {\ {Ahmedabad, IN}}
\begin{cvitems}
  \item Mentored and coordinated a multi-member software team, steering a strategic transition from ad-hoc hardware integration to a modular robotics software stack, which directly enabled a \textbf{national championship win} at DD Robocon 2023 (1st of 63 teams) and representing \textbf{India} at ABU Robocon 2023 (6th of 13 countries).
  \item Implemented inverse kinematics and motion planning for a 3WD omnidirectional mobile base, enabling repeatable point-to-point navigation.
  \item Developed a state estimation framework for a wheeled robot by fusing wheel encoder and IMU data via dead-reckoning, achieving centimeter-level localization accuracy.
\end{cvitems}
%-- Eyantra 2023 --
{\bf Software Member}  \hfill {\ Aug 2022 -- Apr 2023}\\
{\href{https://www.gturoboticsclub.com/acheivements}{Team GRC}}{, }{\href{https://www.youtube.com/watch?v=EAOabSZD77o}{e-Yantra 2023}} \hfill {\ {Ahmedabad, IN}}
\begin{cvitems}
  \item Developed a 3WD holonomic mobile base simulation in ROS and Gazebo, enabling the testing of motion-planning algorithms via high level commands before hardware deployment.
  \item Implemented a low-level control framework using the theory of coordinate transforms and inverse kinematics, converting task-space velocity commands into wheel velocities for hardware deployment.
  \item Built a vision-based pose estimation pipeline with OpenCV and ArUco markers using a ceiling-mounted camera, enabling localization for the for the robot hardware.
  \item Integrated perception, localization, and control into a wheeled robot that autonomously drew digital images on a real-world canvas during the final stage of e-Yantra 2023 (HOLA Bot theme).
\end{cvitems}
%-- Software Member --
{\bf Software Member}  \hfill {\ Oct 2021 -- Jul 2022}\\
{\href{https://www.gturoboticsclub.com/acheivements}{GTU Robotics Club}} \hfill {\ {Ahmedabad, IN}}
\begin{cvitems}
  \item Developed software for a semi-autonomous wheeled robot by fusing data from multiple onboard sensors, enabling robust and low-latency teleoperation during competition runs.
  \item Implemented a real-time perception pipeline using YOLOv5 for object detection and tracking, providing reliable visual feedback of the object's position to the shooting mechanism.
  \item Designed and deployed semi-autonomous control logic for the pick-and-place mechanism, reducing operator workload and contributing to a \textbf{top-10 national finish} (10th out of 43 teams) at DD Robocon 2022.
\end{cvitems}
\end{rSection}

%--------------------------------------------------------------------------------
%    PROJECTS
%-----------------------------------------------------------------------------------------------
% \begin{rSection}{Projects}


%--D3book--
% {\bf Comparison of different $3$D object detection CNN
% architectures}\\
% Explored and implemented various $3$D deep networks like PointNet and Dynamic Graph CNN (DGCNN) on Toyota HSR robot, fine-tuned models on synthetic and real world data; executed simple pick and place tasks.


% {\bf Sampling-based motion planning for manipulators}\\
% Implemented PRM and RRT algorithms in \textbf{MATLAB} for robotic manipulator in the generated configuration space, computing and smoothing the shortest collision-free path from start to goal by removing intermediate nodes.


%--copy and paste this region  if you need more--
% \end{rSection}

%--------------------------------------------------------------------------------
%    Achievements and Awards
%-----------------------------------------------------------------------------------------------
\begin{rSection}{Achievements and Awards} 

{2023\hspace{1em}}{\bf Team (India) received SMC Coporation Award}{, }{\href{https://www.abu.org.my/2023/08/28/toyohashi-university-of-technology-of-japan-wins-abu-robocon-2023/}{ABU Robocon 2023}} \hfill Phnom Penh, KH \vspace{-0.5em}

{2023\hspace{1em}}{\bf Team (GRC 2023) secured 1st place (National Champions)}{, }{\href{https://roboticsindia.live/dd-robocon-2023-match-live-score/}{DD Robocon 2023}} \hfill Delhi, IN \vspace{-0.5em}

{2022\hspace{1em}}{\bf Team (GRC 2022) received Visvesvaraya Best Design Award and a cash prize} \\
{\bf\hphantom{2022}\hspace{1em}of \texteuro 1000}{, }{\href{https://roboticsindia.live/dd-robocon-2022-result/}{DD Robocon 2022}} \hfill Delhi, IN

\end{rSection}

%--------------------------------------------------------------------------------
%    Open source contributions
%-----------------------------------------------------------------------------------------------
\begin{rSection}{Open Source Contributions}
    {\href{https://github.com/shahjesal15/dynamic_reconfigure}{\bf Rviz Dynamic Reconfigure}} \hfill {Maintainer} \\ [0.4em]
    Created a \textbf{RViz2} plugin for GUI-based dynamic parameter reconfiguration, allowing real-time inspection, modification, and logging of \textbf{ROS2 node} parameters, enhancing development productivity.
\end{rSection}

%----------------------------------------------------------------------------------------
%	SKILLS SECTION
%----------------------------------------------------------------------------------------

% \begin{rSection}{Service}
% %--copy and paste this region  if you need more--
% {\textbf {Manuscript Reviewer}, International Communication Association} \hfill {2019} 
% % {\hspace{10em} Divisions: Communication \& Technology, Information Systems}
% \end{rSection}
%----------------------------------------------------------------------------------------
%	SKILLS SECTION
%----------------------------------------------------------------------------------------
% \begin{rSection}{Skills}
% \resumeHeadingListStart{}

% {\bf Programming Languages} \hfill C++, Python, MATLAB, JavaScript, Bash\\[0.35em]
% {\bf Software/Tools} \hfill ROS 2, Linux, CMake, Docker, MoveIt!, Git, Gazebo, Nvidia Isaac Sim, \LaTeX \\[0.35em]
% {\bf Libraries } \hfill PCL, OpenCV, PyTorch, TensorFlow, NumPy, Matplotlib, scikit-learn,  OpenAI Gym, JAX, Eigen, \newline \null \hfill Open3D, OctoMap, SciPy, Drake, CasADi, CVXPY, Numba \\[0.35em]
% {\bf Platforms } \hfill Clearpath, Universal Robots, Franka Emika, Unitree, Human Support Robot, Xarm, Kinnova,\newline \null \hfill TurtleBot, Nvidia Jetson, Raspberry Pi, Arduino, STM32 \\[0.35em]
% {\bf Interests } \hfill Aerospace and Aviation simulation, Piano, Guitar, Books
% \resumeHeadingListEnd{}
% \end{rSection}
\begin{rSection}{Skills}
\resumeHeadingListStart{}
% \vspace{-1em}
\resumeSectionType{Programming Languages}{:}{C++, Python, Bash}
\resumeSectionType{Software/Tools}{:}{ROS 2, Linux, CMake, Git, Gazebo, MuJoCo, Nvidia Isaac Lab, Qt, LaTeX}
\resumeSectionType{Libraries}{:}{PyTorch, TensorFlow, NumPy, Matplotlib, scikit-learn, Eigen, CVXPY, Pinocchio, YOLOv5, OpenCV}
\resumeSectionType{Platforms}{:}{Nvidia Jetson, Raspberry Pi, Arduino, STM32}
\resumeHeadingListEnd{}
\end{rSection}

% \newpage


% \end{rSection}



\newpage
\thispagestyle{reference}


\end{document}----------------------------
